\documentclass[12pt]{article}

\usepackage{fullpage}
\usepackage{graphicx}

\title{Building an Anonymous Community on Tor}
\author{James Lee\thanks{
	Department of Computer Science and Electrical Engineering,
	University of Maryland Baltimore County,
	{\tt jlee23@umbc.edu}}}
\date{\today}

\begin{document}
\maketitle

\paragraph{Keywords} Anonymous Communication, Tor, P2P, Hub, Community

\paragraph{Description} To provide simple means to build communities of anonymous users which may talk and share files with each other through the design and implementation of a Direct Connect-like peer-to-peer (P2P) service that uses Tor for all underlying communication.

\paragraph{Budget} \emph{To be determined...}

\paragraph{Deliverables} \emph{To be determined...}

\section{Executive Summary}
To communicate anonymously with others means that an observer is unable to identify the source, destination, and message of an intercepted packet.  Tor is a software project which anonymizes arbitrary TCP communication by routing packets through multiple nodes before reaching their destination.  It also provides a mechanism to hide services so that they are only accessible by a pseudonym on the Tor network.  However, users must know the pseudonym in order to establish a connection to the hidden service.  Additionally, hidden services are difficult for novice computer users to set up, requiring knowledge of private key management and TCP/IP addressing.  Because of this and the difficulty of discovering hidden services, there don't exist any anonymous communities on the Tor network.

I intend to make the use of Tor to communicate and share files so easy that anonymous communities can begin to grow.  To do this, I will design and implement a hub-based P2P protocol similar to the popular Direct Connect service.  The protocol will require that users communicate over Tor and are identified only by pseudonyms, not IP address like typical P2P protocols.  The client implementation will bundle a Tor client and configure it automatically.  It will also have a basic graphical user interface (GUI) and a one-click installer.

\section{Motivation}
I am proposing this project because I want to be able to participate in a community anonymously.

Why would anyone want to remain anonymous?  Users may want to remain anonymous because they have material that is illegal or for which the distribution is illegal.  The material might just be embarrassing if not illegal.  They may also fear retribution for communicating a sensitive or offensive message.  Anonymous networks provide a great way for users to voice opinions.  They also give a voice to people who are otherwise censored by a governing body.

Typical communication on the Internet is not anonymous.  All users can be identified by their unique IP address.  Anonymous networks such as ``mixes'' as originally proposed by David Chaum in 1981 route encrypted messages through multiple nodes in a network.  As a result, no single node can discover the source, destination, and contents of messages.

Why would someone want to participate in a community?  Communities are collections of users with some shared interest.  All the users have a way to discover all the other users in the community, that way they can meet other people with a shared interest.  In communities where users are identifiable, usually by a nick name, it is possible to build a reputation with the other users, even develop friendships.  Simply put, communities are fun ways of interacting with other people.

How can a community be built?  Most importantly, there must be some way for users to know the other members of the community.  Users could be known could be by nick name or pseudonym.  The users must be easy to discover.  If users can't be discovered, there is no way for someone to find someone else with whom to share their common interest.  Most users will also need some reason to join a community.  Some will expect some sort of content.  A community must also be easy to join.  If it is too dificult, users will not feel it is worth their time.  I am proposing that the ability to communicate and share files anonymously will entice users to join an anonymous community.

Users will have those abilities through the design and implementation of a new P2P protocol similar to Direct Connect.  They will be kept anonymous by Tor.

Why base the design off Direct Connect?  Direct Connect is known for its simplicity.  Users connect to a hub which keeps track of who is online in the community, that way new users are easily discoverable.  Hubs also facilitate searching and chatting since they serve as a single broadcast point to all users.  Also important is that having distinct hubs allow multiple communities to grow, each with special areas of interest.  A system similar to Direct Connect will give users significant incentive to join.

Why use Tor to anonymize communication?  Tor is a well-studied, mature implementation of onion-routing.  Onion-routing routes messages like Chaum's mixes, but on persistent routes, reducing latency and increasing throughput which is ideal for transferring files.  Tor is the largest active anonymity network and is the subject of academic scrutiny, the benefits of which applications built on Tor receive automatically.  Tor is unique in that it can route arbitrary TCP streams.  It provides a standard SOCKS proxy interface so most networking libraries can use Tor unmodified.  Tor also provides a mechanism to hide services so that they are only accessible by a pseudonym on the Tor network.  Clients in my system will be hidden by this mechanism and users will be identified by the pseudonym that Tor provides.

\section{Previous Work}
\emph{To be completed...}

\section{Specific Aims and Plan}

\section{Deliverables}


\section{Issues}

\section{Bibliography}
\emph{To be completed...}

\section{About Me}
\emph{To be completed...}

\section{Schedule}

\section{Budget}
\emph{To be determined...}

\appendix
\section{Research Conference}

\end{document}
