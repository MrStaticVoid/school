\documentclass[12pt]{article}

\usepackage{fullpage}
\usepackage{graphicx}

\title{A Proposal for Simplifying Anonymous P2P Communication Using Tor}
\author{James Lee\thanks{
	Department of Computer Science and Electrical Engineering,
	University of Maryland Baltimore County,
	{\tt jlee23@umbc.edu}}}
\date{\today}

\begin{document}
\maketitle

\paragraph{Keywords} Anonymous Communication, Tor, P2P

\paragraph{Description} To simplify anonymous conversation and file-transfer through the design and implementation of a Direct Connect-like P2P service that uses Tor for all underlying communication.

\paragraph{Budget} \emph{To be determined...}

\paragraph{Deliverables} Hub and client protocol documentation, source code of implementation, simple client installer, final project report, presentation, and slide show.

\section{Executive Summary}
To communicate anonymously with others means that an observer is unable identify the source and destination of an intercepted packet.  Tor is a software project which anonymizes arbitrary TCP communication by routing packets through multiple nodes in a manner originally described by David Chaum.  The purpose of this project is to make the use of Tor transparent in the design of a hub-style chatting and file-sharing application similar to Direct Connect.  The first goal will be to automate the Tor service configuration.  Next a simple protocol will be defined and services will be implemented for client-to-hub and client-to-client communication which must only use pseudonyms and not IP addresses.  Finally, a basic client graphical user interface will be designed.  The results will be that users can continue to have a familiar P2P experience while reducing their vulnerability to traffic-analysis.

\section{Motivation}
In a P2P system, two peers who communicate with each other must know the other's address to establish the connection.  Over the Internet, this address is a unique IP address assigned to a customer by his/her ISP.  Users who wish to exercise their fair-use rights to a copyrighted work by downloading it over a P2P network face the fear of litigation from the copyright holder, who can find their identity by contacting their ISP.  For example I downloaded a couple of minutes of a movie for review using a popular P2P network.  I did not know I was downloading it from the copyright holder.  They contacted my ISP who then sent threats to me.  I was able to clear everything up with my ISP and the copyright holder because I was able to prove the amount that I downloaded was well within my fair-use rights, but others are not so lucky.  That was when I began investigating anonymous communication.

Today, Tor is the most active and largest anonymous communication network.  It routes encrypted packets through multiple nodes in a network as proposed by David Chaum in his 1981 seminal paper so that no single node can know both the source and destination of the route.  Tor is unique in that it supports anonymizing arbitrary TCP streams using its built-in SOCKS interface.  This means that some existing P2P systems can take advantage of the Tor network.  There are several problems with this, though.

For peers to know where other peers are located, most protocols have some system for peers to announce who they are.  In BitTorrent, for example, peers identify themselves with a central ``tracker.''  Since these protocols weren't designed to use the Tor network, they identify themselves using their IP address.  So even if they use Tor, peers' identities can be discovered.

Additionally, it adds another layer of complication to the scenario since two programs need to be used: the P2P program itself and the Tor client.  That means more confusion for typical users.

My solution is to develop a system where Tor can be automatically configured and run in the background when a user wishes to engage in P2P communication.  Each peer's Tor client would be configured to operate as a hidden server which would give them all pseudonyms that they can be identified by.  Then a new protocol will be developed which only knows about clients' pseudonyms.  This protocol will be designed after Direct Connect.

Direct Connect is a simple P2P system which revolves around a central hub.  This hub only keeps track of who is logged in, and in our case, it will just hold pseudonyms.  When a user logs in to the hub, they are able to broadcast messages to all the users of the hub, basically enabling chat, or further establish links with other users, with whom they can talk one-on-one or transfer files.

Users of P2P systems are often computer novices, so this system should not appear any different than existing P2P systems.

Finally, having a distinct network specifically for file-sharing will provide a tool to help analyze and improve the performance of Tor.

\section{Previous Work}
\emph{To be completed...}

\section{Specific Aims and Plan}
First, Tor clients must be configured and run upon request.  A Java library will be created to create a configuration file and run the Tor binary.  The Tor binary can be stored within the library since Java libraries are simple zip files.  It can be self-extracted at runtime.  Further configuration and status monitoring of Tor can be accomplished through its controller protocol as defined at \verb+https://www.torproject.org/svn/torctl/doc/howto.txt+.

Next, a hub protocol must be developed.  It must handle incoming connections and remember who is connected.  It must send notifications to all the users when a new user logs on.  It must accept incoming chat messages and distribute them to all the connected nodes.  It must pass on search requests to all the nodes, and pass back the search results to the requester.  Finally it must handle disconnection of nodes.

A client protocol must be written to conform to what the hub expects.  Additionally, clients will need to be able to send text chat messages to each other, request file lists, and files themselves.

The clients must keep track of files they own.  They must recognize when a file has changed, or when a file has been added or removed.

All implementations will be written in Java since it is easily portable.  The client will be installable with a single click using Java Web Start.

\section{Deliverables}
The hub and client protocols will be formally documented.  The source code of the implementation of the hub and the client will be provided.  A Java Web Start installer will be created.  A final report, presentation, and slide show will be created.

\section{Issues}
The must difficult issue that I foresee is how to realistically test a P2P system.  It would be impractical to actually install the client on several computers.  Fortunately Linux provides means to create virtual network interfaces.  Using that capability, scripts can be written that bind multiple instances of the P2P client to different virtual network interfaces.

\section{Bibliography}
\emph{To be completed...}

\section{About Me}
I am a senior Computer Science student at the University of Maryland Baltimore County.  My primary interest is in open-source desktop software.  I love making complicated tasks easier as exemplified by my program iriverter (\verb+http://iriverter.thestaticvoid.org+), which is a Java frontend to mencoder to convert videos.

\section{Schedule}
\begin{center}
\begin{tabular}{ll}
\multicolumn{1}{c}{\bf Date} & \multicolumn{1}{c}{\bf Milestone}\\
\hline
March 9 & Tor configuration library complete\\
March 16 & Protocols defined and documented\\
March 23 & Hub completed\\
March 30 & Client completed\\
April 6 & User interface and installer completed
\end{tabular}
\end{center}

\section{Budget}
\emph{To be determined...}

\appendix
\section{Research Conference}
Tor was introduced at the 13th USENIX Security Symposium.

\end{document}
