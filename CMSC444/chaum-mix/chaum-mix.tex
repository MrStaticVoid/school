\documentclass[default,pdf,colorBG,slideColor]{prosper}
\hypersetup{pdfpagemode=FullScreen}

\title{Anonymous Mixes}
\subtitle{As Proposed by David Chaum}
\author{James Lee}
\email{jlee23@umbc.edu}
\institution{University of Maryland, Baltimore County}

\begin{document}
\maketitle

\begin{slide}{Outline}
\begin{itemize}
\item Introduction
\item Public Key Cryptography
\item Assumptions
\item Mixes
\item Return Addresses
\item Digital Pseudonyms
\item Conclusion
\item Questions
\item References
\end{itemize}
\end{slide}

\begin{slide}{``The Traffic Analysis Problem''}
\begin{itemize}
\item Anyone along a physical circuit can watch traffic flow.
\item Trying to keep confidential who converses with whom and when.
\item Solution based on public key cryptography.
\item Previous work required common authority.
\item Now all participants are authorities.
\end{itemize}
\end{slide}

\begin{slide}{Public Key Cryptography}
\begin{itemize}
\item Solves the key distribution problem.
\item User creates two keys: public $K$ and private $K^{-1}$.
\item Public and private keys are inverses of each other.
\end{itemize}
\end{slide}

\begin{slide}{Sealing a Message}
\begin{itemize}
\item A message is \emph{sealed} by any user so that only a certain user can decrypt it.
\item $K(R, X)=Y$
\item Append random bits $R$ before message $X$ to make brute-force attacks harder.
\item Designated user can decrypt using their private key: $K^{-1}(Y)=R, X$.
\end{itemize}
\end{slide}

\begin{slide}{Signing a Message}
\begin{itemize}
\item A message is \emph{signed} so that any user can verify its source.
\item $K^{-1}(C, X)=Y$
\item Any user can decrypt using the public key of the claimed source: $K(Y)=C, X$.
\item If $C$ matches expected value, then $X$ is actually from the claimed source.
\end{itemize}
\end{slide}

\begin{slide}{Assumptions}
\begin{enumerate}
\item The cryptographic algorithm is strong.
\item Anyone may watch the underlying communication system.  Anyone may interfere.
\end{enumerate}
\end{slide}

\begin{slide}{Mixes}
\begin{itemize}
\item A \emph{mix} is a computer which accepts and delivers messages.
\item A user wishing to remain anonymous to the recipient will send a message through the mix.
\item Every participant in the system has a known public key.
\item User seals message for the recipient, then seals the sealed message with the recipient's address for the mix.
\item Mix will send the original sealed message to the designated address.
\item $K_{mix}(R_{mix}, K_{dest}(R_{dest}, M), A)\rightarrow K_{dest}(R_{dest}, M), A$
\end{itemize}
\end{slide}

\begin{slide}{Mixes (cont.)}
\begin{itemize}
\item Purpose of mixes is to hide correspondences between the items in the input and output.
\item Order of arrival is hidden by outputting messages in uniformly sized, lexicographically ordered batches.
\item Special care must be taken to make sure duplicate messages are never sent twice.
\end{itemize}
\end{slide}

\begin{slide}{Signed Receipts}
\begin{itemize}
\item How does a user know if the mix discarded the message?
\item Mix sends the signed batch back to the user.
\item User checks the signature.
\item Checks if original message exists in the batch.
\end{itemize}
\end{slide}

\begin{slide}{Cascades}
\begin{itemize}
\item Series of mixes.
\item Prevents any single mix from knowing both the source and destination of a message.
\item $K_{mix_1}(R_{mix_1}, K_{mix_2}(R_{mix_2}, K_{dest}(R_{dest}, M), A), A_{mix_2})$
\end{itemize}
\end{slide}

\begin{slide}{Return Addresses}
\begin{itemize}
\item Alice can form an untraceable return address: $K_{mix}(R_{mix}, A_{alice}), K_{alice}$.
\item Alice can send the address to Bob in an anonymous message.
\item For Bob to send a reply, he forms a message for a mix like: $K_{mix}(R_{mix}, A_{alice}), K_{alice}(R_{bob}, M)$.
\end{itemize}
\end{slide}

\begin{slide}{Digital Pseudonyms}
\begin{itemize}
\item Public key used to identify a user.
\item A authority holds a \emph{roster} of trusted pseudonyms.
\item Applicant sends his public key, $K$, to the authority through a mix.
\item Example: Registered voters can submit a signed, anonymous ballot $K_{mix}(R_{mix}, K, K^{-1}(C, Vote))$.
\item Recipient of ballot checks $K$ against roster and records the vote.
\end{itemize}
\end{slide}

\begin{slide}{Questions}

\end{slide}

\begin{slide}{References}
\begin{itemize}
\item David Chaum. Untraceable electronic mail, return addresses, and digital pseudonyms. In \emph{Communications of the ACM 4(2)}, February 1981.
\item George Danezis and Claudia Diaz. A Survey of Anonymous Communication Channels. Submitted to the \emph{Journal of Privacy Technology}, 40 pages, 2006.
\end{itemize}
\end{slide}
\end{document}
